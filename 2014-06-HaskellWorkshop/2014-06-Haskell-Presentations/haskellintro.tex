\documentclass{beamer}

\mode<presentation> {
\usetheme{Madrid}
\setbeamertemplate{navigation symbols}{}
}

\usepackage{graphicx}
\usepackage{booktabs}
\usepackage[T1]{fontenc}
\usepackage{minted}
\usepackage{upquote}
\AtBeginDocument{
\def\PYZsq{\textquotesingle}
}

\title[Haskell Intro]{Haskell Introduction}

\author{Jean-Pierre Rupp}
\institute[Haskoin]
{
Haskoin Limited \\
\medskip
\textit{root@haskoin.com}
}
\date{\today}

\begin{document}

\begin{frame}
\titlepage
\end{frame}

\begin{frame}
\frametitle{Haskell Qualities}
\begin{itemize}
\item Referential transparency
\item Lazy evaluation
\item Type inference
\item Immutable data structures
\item Pattern Matching
\item Property testing
\item Made by really smart people (with PhDs)
\end{itemize}
\end{frame}


\begin{frame}[fragile]
\frametitle{Functions \& Lists}
\begin{minted}{haskell}
-- Functions
square x = x * x

-- Lists
smallNums = [1,2,3,4]
oneMillion = [1..1000000]
nats = [1..]
helloWorld = "hello world!"
abc = 'a':'b':'c':[]
\end{minted}
\end{frame}


\begin{frame}[fragile]
\frametitle{Currying}
\begin{minted}{haskell}
-- Functions with multiple parameters
prod x y = x * y

-- Partially-applied functions
multBy4 = prod 4
addOne = (+1)
prodShort = (*)
square = (^2)

-- Anonymous functions
addTwo = \x -> x + 2
\end{minted}
\end{frame}


\begin{frame}[fragile]
\frametitle{First-Class Functions}
\begin{minted}{haskell}
-- Apply a function to all elements in list
map (+1) [1..10]         -- Add one
take 5 (map (^2) [1..])  -- Take five squared integers

-- Other list operations
sort "daynet"            -- "adenty"
filter (>10) [5..15]     -- [11,12,13,14,15]
concat ["ad","en","ty"]  -- "adenty"
\end{minted}
\end{frame}


\begin{frame}[fragile]
\frametitle{Types}
\begin{minted}{haskell}
-- Simple type
data Soca = OneCent | FiveCent | TenCent | Dollar

-- Polymorphic parametrized algebraic data type
data List a = Nil | Cons a (List a)
Cons 'a' (Cons 'b' (Cons 'c' Nil))   -- Equivalent to "abc"

-- Type alias (and some pattern matching)
type Complex = (Double, Double)
cpxProd :: Complex -> Complex -> Complex
cpxProd (a,b) (c,d) = ((a*c-b*d),(b*c+a*d))
\end{minted}
\end{frame}


\begin{frame}[fragile]
\frametitle{Parametrized Types}
\begin{minted}{haskell}
-- Find smallest
data Soca = OneCent | FiveCent | TenCent | Dollar
    deriving Ord

smallest :: Ord a => [a] -> a
smallest [] = undefined
smallest [x] = x
smallest (x:xs) = min x (smallest xs)

smallest [TenCent, FiveCent, Dollar, OneCent] -- OneCent
smallest [5,3,10,9]  -- 3
smallest "adenty"    -- 'a'
\end{minted}
\end{frame}


\begin{frame}[fragile]
\frametitle{Monads}
\begin{minted}{haskell}
-- Maybe monad
safeDiv :: Integral a => a -> a -> Maybe a
safeDiv x y = guard (y /= 0) >> return (x `div` y)

-- IO monad
main = do
    putStrLn "What is your name?"
    name <- getLine
    putStrLn ("Nice meeting you " ++ name ++ ".")
\end{minted}
\end{frame}


\end{document}